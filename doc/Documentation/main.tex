%%%%%%%%%%%%%%%%%%%%%%%%%%%%%%%%%%%%%%%%%
% The Legrand Orange Book
% LaTeX Template
% Version 1.4 (12/4/14)
%
% This template has been downloaded from:
% http://www.LaTeXTemplates.com
% Original author:
% Mathias Legrand (legrand.mathias@gmail.com)
% License:
% CC BY-NC-SA 3.0 (http://creativecommons.org/licenses/by-nc-sa/3.0/)
%
% Important note:
% Chapter heading images should have a 2:1 width:height ratio,
% e.g. 920px width and 460px height.
%
%%%%%%%%%%%%%%%%%%%%%%%%%%%%%%%%%%%%%%%%%

%----------------------------------------------------------------------------------------
%	PACKAGES AND OTHER DOCUMENT CONFIGURATIONS
%----------------------------------------------------------------------------------------

\documentclass[11pt,fleqn]{book} % Default font size and left-justified equations

\usepackage[top=3cm,bottom=3cm,left=3.2cm,right=3.2cm,headsep=10pt,a4paper]{geometry} % Page margins

\usepackage{xcolor} % Required for specifying colors by name
\definecolor{ocre}{RGB}{243,102,25} % Define the orange color used for highlighting throughout the book

% Font Settings
\usepackage{avant} % Use the Avantgarde font for headings
%\usepackage{times} % Use the Times font for headings
\usepackage{mathptmx} % Use the Adobe Times Roman as the default text font together with math symbols from the Sym­bol, Chancery and Com­puter Modern fonts

\usepackage{microtype} % Slightly tweak font spacing for aesthetics
\usepackage[utf8]{inputenc} % Required for including letters with accents
\usepackage[T1]{fontenc} % Use 8-bit encoding that has 256 glyphs

% Bibliography
%\usepackage[style=alphabetic,sorting=nyt,sortcites=true,autopunct=true,babel=hyphen,hyperref=true,abbreviate=false,backref=true,backend=biber]{biblatex}
%\addbibresource{bibliography.bib} % BibTeX bibliography file
%\defbibheading{bibempty}{}

% Index
%\usepackage{calc} % For simpler calculation - used for spacing the index letter headings correctly
%\usepackage{makeidx} % Required to make an index
%\makeindex % Tells LaTeX to create the files required for indexing

\usepackage{todonotes}

\newcommand{\ProjectTitle}{isySUR}
\newcommand{\pt}{\ProjectTitle}
\newcommand{\todol}{\todo[inline]}

\usepackage{scrextend} %used for footref - repetition of footnotes

%----------------------------------------------------------------------------------------

\input{structure} % Insert the commands.tex file which contains the majority of the structure behind the template

\begin{document}

%----------------------------------------------------------------------------------------
%	TITLE PAGE
%----------------------------------------------------------------------------------------

\begingroup
\thispagestyle{empty}
\AddToShipoutPicture*{\put(6,5){\includegraphics[scale=1]{bg}}} % Image background
\centering
\vspace*{9cm}
\todol{Create a nice background image}
\par\normalfont\fontsize{35}{35}\sffamily\selectfont
\includegraphics[width=\textwidth]{LogoWithName.jpg}
%\ProjectTitle\par % Book title
\vspace*{1cm}
{\Huge Manual}\par
\endgroup

%----------------------------------------------------------------------------------------
%	COPYRIGHT PAGE
%----------------------------------------------------------------------------------------

%\newpage
%~\vfill
%\thispagestyle{empty}

%\noindent Copyright \copyright\ 2013 John Smith\\ % Copyright notice

%\noindent \textsc{Published by Publisher}\\ % Publisher

%\noindent \textsc{book-website.com}\\ % URL

%\noindent Licensed under the Creative Commons Attribution-NonCommercial 3.0 Unported License (the ``License''). You may not use this file except in compliance with the License. You may obtain a copy of the License at \url{http://creativecommons.org/licenses/by-nc/3.0}. Unless required by applicable law or agreed to in writing, software distributed under the License is distributed on an \textsc{``as is'' basis, without warranties or conditions of any kind}, either express or implied. See the License for the specific language governing permissions and limitations under the License.\\ % License information

%\noindent \textit{First printing, March 2013} % Printing/edition date

%----------------------------------------------------------------------------------------
%	TABLE OF CONTENTS
%----------------------------------------------------------------------------------------

\chapterimage{Kollage.jpg} % Table of contents heading image

\pagestyle{empty} % No headers

\tableofcontents % Print the table of contents itself

\cleardoublepage % Forces the first chapter to start on an odd page so it's on the right

\pagestyle{fancy} % Print headers again

%----------------------------------------------------------------------------------------
%	Introduction
%----------------------------------------------------------------------------------------

\chapterimage{Kollage.jpg}
\chapter{Introduction to \ProjectTitle}
Welcome to isySUR (pronounce isy as ['i:zi]), a tool helping you to find the sphere of influence of a space usage rule. For starting the command line version please see section \ref{sec:QuickstartTerminal}. For using the window version please read section \ref{sec:QuickstartWindow}. You can also run the window version on your android device. This is described in section \ref{sec:android}. For more details about the program see section \ref{sec:whatFor}.

\section{Quick start}
First decide if you like to use the command line version of \pt\ or if you prefer a graphical user interface. Please proceed with section \ref{sec:QuickstartTerminal} or section \ref{sec:QuickstartWindow} depending on your choice.

\subsection{\ProjectTitle\ in command line version}\label{sec:QuickstartTerminal}
\todol{Explain how to quick start}

\subsection{\ProjectTitle\ in window version}\label{sec:QuickstartWindow}
\todol{Explain how to quick start}

\section{What is \ProjectTitle?}\label{sec:whatFor}
\pt is a program that calculates the sphere of influence of a space usage rule by its coordinates. It was developed within the informatiCup, a programming competition for students organized by the German Informatics Society (Gesellschaft für Informatik e.V.). Our team is called isy and consists of Adriana-Victoria Dreyer, Jacqueline Hemminghaus, Jan Pöppel and Thorsten Schodde, four students of the master study course `intelligent systems' at Bielefeld University.

\todol{using OSM}
\todol{Features: why to use this software}

%----------------------------------------------------------------------------------------
%	Using the program
%----------------------------------------------------------------------------------------

\chapterimage{Kollage.jpg}
\chapter{Using \ProjectTitle}
In this chapter the usage of \pt\ is described in detail. First you can find installation instructions in section \ref{sec:installation}, detailed usage instruction for both program versions are following in section \ref{sec:usage}.

\section{Requirements and installation}\label{sec:installation}
\pt\ is written in Python so you get a Python script, which is the main program, and a package with the tools. For running the command line version you need not much more than basic Python. If you prefer to use a graphical user interface you will have to install some more packages.
\subsection{Requirements}
\begin{enumerate}
	\item \textbf{requirements for the command line version:}
	\begin{itemize}
		\item Python 2.7 \todo{3 auch ok?} %Thorsten: Maybe importent: 32/64 bit? - Is this important for us? As far as I know it works for both.
		\item requests (HTTP library)
		\item internet connection
	\end{itemize}
	~\\
	To run a Python script you need Python. Python comes with most Linux distributions. For the use on Windows machines you will have to install Python manually\footnote{get Python here: https://www.python.org}. \pt\ uses data from OpenStreetMap\footnote{http://www.openstreetmap.org} to calculate the sphere of influence. Therefore an internet connection is required. The data transfer in Python is realised with the requests\footnote{http://docs.python-requests.org\label{fn:requests}} library. You can easily install it using pip\footnote{https://pip.pypa.io}, a Python installation tool
	\begin{verbatim}
	$ pip install requests
	\end{verbatim}
	or see requests web page\footref{fn:requests}.
	\item \textbf{additional requirements for the GUI version:}
	\begin{itemize}
		\item \todol{requirements for GUI}
	\end{itemize}
\end{enumerate}
\subsection{Installation}
\todol{Unterschiede bei GUI?} %Thorsten: Most packages are published as .zip for windows too. do we need it? - No, no zip, we publish ONE archiv. Windows user can also unpack .tar.gz
You got an archive of type .tar.gz. Please unpack the archive into a directory you want. Because \pt\ is a Python script you do not need to install it. It is also possible to install only the dependencies and run \pt\ through typing:
\begin{verbatim}
	$ python run_isySUR.py
\end{verbatim}
~\\

For installing the script and tool package browse the directory you chose and install \pt\ with help of Python:
\begin{verbatim}
	$ python setup.py install
\end{verbatim}
Now you are able to start \pt\ anytime through
\begin{verbatim}
	$ run_isySUR.py
\end{verbatim}

\section{Command line and GUI version}\label{sec:usage}
\pt\ comes with two version: One for command line use and another one for users who prefer to use a graphical user interface.
\subsection{Command line version}
\subsubsection{Parameters}
\todol{what parameters and how to use it}
\subsection{GUI version}
\subsection{Using android}\label{sec:android}



%----------------------------------------------------------------------------------------
%	Technical and implementation details
%----------------------------------------------------------------------------------------

\chapterimage{Kollage.jpg}
\chapter{Technical and implementation details}
\todol{create sections}
\section{}

%----------------------------------------------------------------------------------------
%	List of TODOS
%----------------------------------------------------------------------------------------

\listoftodos

%========================================================================================
%   EVERYTHINGS BELOW IS EXAMPLE TEXT - TO BE DELEATED IN THE END
%   (nice examples how to use some commands)
%========================================================================================

%----------------------------------------------------------------------------------------
%	CHAPTER 1
%----------------------------------------------------------------------------------------

\chapterimage{chapter_head_2.pdf} % Chapter heading image

\chapter{EXAMPLE Chapter}

\section{Paragraphs of Text}\index{Paragraphs of Text}

\lipsum[1-7] % Dummy text

%------------------------------------------------

\section{Citation}\index{Citation}

This statement requires citation \cite{book_key}; this one is more specific \cite[122]{article_key}.

%------------------------------------------------

\section{Lists}\index{Lists}

Lists are useful to present information in a concise and/or ordered way\footnote{Footnote example...}.

\subsection{Numbered List}\index{Lists!Numbered List}

\begin{enumerate}
\item The first item
\item The second item
\item The third item
\end{enumerate}

\subsection{Bullet Points}\index{Lists!Bullet Points}

\begin{itemize}
\item The first item
\item The second item
\item The third item
\end{itemize}

\subsection{Descriptions and Definitions}\index{Lists!Descriptions and Definitions}

\begin{description}
\item[Name] Description
\item[Word] Definition
\item[Comment] Elaboration
\end{description}

%----------------------------------------------------------------------------------------
%	CHAPTER 2
%----------------------------------------------------------------------------------------

%\chapter{In-text Elements}
%
%\section{Theorems}\index{Theorems}
%
%This is an example of theorems.
%
%\subsection{Several equations}\index{Theorems!Several Equations}
%This is a theorem consisting of several equations.
%
%\begin{theorem}[Name of the theorem]
%In $E=\mathbb{R}^n$ all norms are equivalent. It has the properties:
%\begin{align}
%& \big| ||\mathbf{x}|| - ||\mathbf{y}|| \big|\leq || \mathbf{x}- \mathbf{y}||\\
%&  ||\sum_{i=1}^n\mathbf{x}_i||\leq \sum_{i=1}^n||\mathbf{x}_i||\quad\text{where $n$ is a finite integer}
%\end{align}
%\end{theorem}
%
%\subsection{Single Line}\index{Theorems!Single Line}
%This is a theorem consisting of just one line.
%
%\begin{theorem}
%A set $\mathcal{D}(G)$ in dense in $L^2(G)$, $|\cdot|_0$. 
%\end{theorem}

%------------------------------------------------

%\section{Definitions}\index{Definitions}
%
%This is an example of a definition. A definition could be mathematical or it could define a concept.
%
%\begin{definition}[Definition name]
%Given a vector space $E$, a norm on $E$ is an application, denoted $||\cdot||$, $E$ in $\mathbb{R}^+=[0,+\infty[$ such that:
%\begin{align}
%& ||\mathbf{x}||=0\ \Rightarrow\ \mathbf{x}=\mathbf{0}\\
%& ||\lambda \mathbf{x}||=|\lambda|\cdot ||\mathbf{x}||\\
%& ||\mathbf{x}+\mathbf{y}||\leq ||\mathbf{x}||+||\mathbf{y}||
%\end{align}
%\end{definition}

%------------------------------------------------

%\section{Notations}\index{Notations}
%
%\begin{notation}
%Given an open subset $G$ of $\mathbb{R}^n$, the set of functions $\varphi$ are:
%\begin{enumerate}
%\item Bounded support $G$;
%\item Infinitely differentiable;
%\end{enumerate}
%a vector space is denoted by $\mathcal{D}(G)$. 
%\end{notation}

%------------------------------------------------

%\section{Remarks}\index{Remarks}
%
%This is an example of a remark.
%
%\begin{remark}
%The concepts presented here are now in conventional employment in mathematics. Vector spaces are taken over the field $\mathbb{K}=\mathbb{R}$, however, established properties are easily extended to $\mathbb{K}=\mathbb{C}$.
%\end{remark}

%------------------------------------------------

%\section{Corollaries}\index{Corollaries}
%
%This is an example of a corollary.
%
%\begin{corollary}[Corollary name]
%The concepts presented here are now in conventional employment in mathematics. Vector spaces are taken over the field $\mathbb{K}=\mathbb{R}$, however, established properties are easily extended to $\mathbb{K}=\mathbb{C}$.
%\end{corollary}

%------------------------------------------------

%\section{Propositions}\index{Propositions}
%
%This is an example of propositions.
%
%\subsection{Several equations}\index{Propositions!Several Equations}
%
%\begin{proposition}[Proposition name]
%It has the properties:
%\begin{align}
%& \big| ||\mathbf{x}|| - ||\mathbf{y}|| \big|\leq || \mathbf{x}- \mathbf{y}||\\
%&  ||\sum_{i=1}^n\mathbf{x}_i||\leq \sum_{i=1}^n||\mathbf{x}_i||\quad\text{where $n$ is a finite integer}
%\end{align}
%\end{proposition}
%
%\subsection{Single Line}\index{Propositions!Single Line}
%
%\begin{proposition} 
%Let $f,g\in L^2(G)$; if $\forall \varphi\in\mathcal{D}(G)$, $(f,\varphi)_0=(g,\varphi)_0$ then $f = g$. 
%\end{proposition}

%------------------------------------------------

%\section{Examples}\index{Examples}
%
%This is an example of examples.
%
%\subsection{Equation and Text}\index{Examples!Equation and Text}
%
%\begin{example}
%Let $G=\{x\in\mathbb{R}^2:|x|<3\}$ and denoted by: $x^0=(1,1)$; consider the function:
%\begin{equation}
%f(x)=\left\{\begin{aligned} & \mathrm{e}^{|x|} & & \text{si $|x-x^0|\leq 1/2$}\\
%& 0 & & \text{si $|x-x^0|> 1/2$}\end{aligned}\right.
%\end{equation}
%The function $f$ has bounded support, we can take $A=\{x\in\mathbb{R}^2:|x-x^0|\leq 1/2+\epsilon\}$ for all $\epsilon\in\intoo{0}{5/2-\sqrt{2}}$.
%\end{example}
%
%\subsection{Paragraph of Text}\index{Examples!Paragraph of Text}
%
%\begin{example}[Example name]
%\lipsum[2]
%\end{example}

%------------------------------------------------

%\section{Exercises}\index{Exercises}
%
%This is an example of an exercise.
%
%\begin{exercise}
%This is a good place to ask a question to test learning progress or further cement ideas into students' minds.
%\end{exercise}

%------------------------------------------------

%\section{Problems}\index{Problems}
%
%\begin{problem}
%What is the average airspeed velocity of an unladen swallow?
%\end{problem}

%------------------------------------------------

%\section{Vocabulary}\index{Vocabulary}
%
%Define a word to improve a students' vocabulary.
%
%\begin{vocabulary}[Word]
%Definition of word.
%\end{vocabulary}

%----------------------------------------------------------------------------------------
%	CHAPTER 3
%----------------------------------------------------------------------------------------

%\chapterimage{chapter_head_1.pdf} % Chapter heading image
%
%\chapter{Presenting Information}
%
%\section{Table}\index{Table}
%
%\begin{table}[h]
%\centering
%\begin{tabular}{l l l}
%\toprule
%\textbf{Treatments} & \textbf{Response 1} & \textbf{Response 2}\\
%\midrule
%Treatment 1 & 0.0003262 & 0.562 \\
%Treatment 2 & 0.0015681 & 0.910 \\
%Treatment 3 & 0.0009271 & 0.296 \\
%\bottomrule
%\end{tabular}
%\caption{Table caption}
%\end{table}

%------------------------------------------------

%\section{Figure}\index{Figure}
%
%\begin{figure}[h]
%\centering\includegraphics[scale=0.5]{placeholder}
%\caption{Figure caption}
%\end{figure}

%----------------------------------------------------------------------------------------
%	BIBLIOGRAPHY
%----------------------------------------------------------------------------------------

%\chapter*{Bibliography}
%\addcontentsline{toc}{chapter}{\textcolor{ocre}{Bibliography}}
%\section*{Books}
%\addcontentsline{toc}{section}{Books}
%\printbibliography[heading=bibempty,type=book]
%\section*{Articles}
%\addcontentsline{toc}{section}{Articles}
%\printbibliography[heading=bibempty,type=article]

%----------------------------------------------------------------------------------------
%	INDEX
%----------------------------------------------------------------------------------------

%\cleardoublepage
%\phantomsection
%\setlength{\columnsep}{0.75cm}
%\addcontentsline{toc}{chapter}{\textcolor{ocre}{Index}}
%\printindex

%-------------------------------------------------------------------------
